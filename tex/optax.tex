\documentclass[11pt]{article}
\usepackage{amsmath}
\title{A simple optimization problem using OPTAX}
\date{\today}
\author{S S Thakar}
\begin{document}
\maketitle
\section{Problem Statement}
Given a force $F$:
\begin{equation}
  \begin{split}
    F = K(x)x + \frac{\partial K}{\partial x}x, \\
    K(x) = Ax^2 + Bx +C
    \label{eq:problem_statement}
    \end{split}
\end{equation}
subject to the following constraints:
\begin{equation}
  F=F_0=10, \ \text{when} \ x=x_0=0.2
    \label{eq:constraints}
\end{equation}
Find the values of $A,B, \ \text{and} \ C$ such that Eq.\ref{eq:problem_statement} holds true for Eq.\ref{eq:constraints}.
\section{Solution}
We begin by converting the given problem into an optimization problem. Let us define the parameters of interest as $\boldsymbol{\theta}=\{A,B,C\}$. 
Further let us define the loss function as a mean squared error based on the constraints given by Eq.\ref{eq:constraints}:
\begin{equation}
  L(\boldsymbol{\theta}) = \left( F_0 - \left(F(\boldsymbol{\theta,x_0})x_0 + \left.\frac{\partial F(\boldsymbol{\theta},x)}{\partial x}\right|_{x=x_0}\right)\right)^2.
    \label{eq:loss_function}
\end{equation}
Now our problem has been converted to an optimization problem. From an intial guess for $\boldsymbol{\theta}$ we can use gradient descent and/or its variations to minimize $L(\boldsymbol{\theta})$.
\subsection{Gradient descent}
\subsection{Code}
\subsection{Results}
\end{document}
